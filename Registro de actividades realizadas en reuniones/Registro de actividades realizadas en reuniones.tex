% tipo de documento
\documentclass[12pt]{article}

 % librerías
\usepackage{titleps}
\usepackage{fancyhdr}
\usepackage{graphicx}
\usepackage[a4paper]{geometry}
\usepackage[utf8]{inputenc}
 \usepackage{xcolor}  
 \usepackage{lipsum}
% márgenes
\geometry{top=2.5cm, bottom=2.5cm, left=2.0cm, right=2.0cm}

% PRINCIPAL
\begin{document}
%portada
\begin {titlepage}
\centering
\pagecolor{white}
\vspace*{1cm}
{\bfseries\Huge Proyectos Informáticos \par}
{\scshape\huge Registro de actividades realizadas en las reuniones de coordinación \par}
\vspace*{2cm}
\begin{figure}[h]
\begin{center}
\includegraphics[scale=.4]{logo_uca.jpg}
\end{center}
\end{figure}

\vfill
{\large María Segura Bolaños, Abel García Pascual, Alberto Valderas González, Daniela Iossa López, Carlos Vidal Rodríguez, Pablo Reyes Torrejón.\par}
{\large \textbf{2022-2023} \par}
\end{titlepage}


%Contenido
\section{Viernes, 14 de octubre de 2022}
\large Hoy hemos tenido nuestra primera reunión, la cual ha sido un \emph{brainstorming} de ideas sobre las que realizar nuestra aplicación. Finalmente hemos estado estudiando las necesidades públicas de hoy en día, y también hemos estudiado la originalidad de nuestro proyecto, así que hemos decidido orientarlo al tema de actividades deportivas en el gimnasio y recetas saludables que podemos encontrar en la provincia de Cádiz, preferiblemente. \par

\section{Martes, 18 de octubre de 2022}
\large En el día de hoy hemos repartido el trabajo. Nos hemos dedicado a crear los roles necesarios para la llevada a cabo del proyecto integrado y nos los hemos asignado \newline Finalmente, María, Pablo y Daniela se encargarán de la parte de front-end (programación HTML, CSS y Javascript) y Abel, Alberto y Carlos a la parte de back-end. Además, por el momento, Daniela se encargará del diseño gráfico del logo.

\section{Miércoles, 19 de octubre de 2022}
\large Hoy el profesor nos ha dado el \emph{visto bueno} a nuestro proyecto y hemos empezado a documentar nuestras reuniones.
\end{document}